\documentclass{bioinfo}
\copyrightyear{2016} \pubyear{2016}

%% Some pieces required from the pandoc template
\providecommand{\tightlist}{%
  \setlength{\itemsep}{0pt}\setlength{\parskip}{0pt}}


% Pandoc citation processing


% hyperref makes the margins screwy.
% https://groups.google.com/forum/#!topic/latexusersgroup/4W_SwGk6zx4
% http://ansuz.sooke.bc.ca/software/latex-tricks.php
% \usepackage[colorlinks=true, allcolors=blue]{hyperref}

\access{Advance Access Publication Date:   }
\appnotes{Manuscript Category}

\begin{document}
\firstpage{1}

\subtitle{Application Note}

\title[JBrowseR: R Interface to JBrowse 2]{JBrowseR: An R Interface to
the JBrowse 2 Genome Browser}

\author[FirstAuthorLastName \textit{et~al}.]{
Elliot Hershberg\,\textsuperscript{1},
Garrett Stevens\,\textsuperscript{1},
Colin Diesh\,\textsuperscript{1},
Peter Xie\,\textsuperscript{1},
Teresa De Jesus Martinez\,\textsuperscript{1},
Rob Buels\,\textsuperscript{1},
Lincoln Stein\,\textsuperscript{2},
Ian Holmes\,\textsuperscript{1*},
}

\address{
\textsuperscript{1}Department of Bioengineering, University of
California, Berkeley, Berkeley, CA 94720, USA\\
\textsuperscript{2}Ontario Institute for Cancer Research, Toronto, ON
M5G 0A3, Canada\\
}

\corresp{*To whom correspondence should be addressed. E-mail:
ihh@berkeley.edu}

\history{Received on XXX; revised on XXX; accepted on XXX}

\editor{Associate Editor: XXX}

\abstract{
\textbf{Motivation:} Genome browsers are an essential tool in genome
analysis. Modern genome browsers enable complex and interactive
visualization of a wide variety of genomic data modalities. While such
browsers are very powerful, they can be challenging to configure and
program for bioinformaticians lacking expertise in web development.\\
\textbf{Results:} We have developed an R package that provides an
interface to the JBrowse 2 genome browser. The package can be used to
configure and customize the browser entirely with R code. The browser
can be deployed from the R console, or embedded in Shiny applications or
R Markdown documents.\\
\textbf{Availability:} JBrowseR is available for download from CRAN, and
the source code is openly available from the Github repository at\\
\textbf{Contact:}ihh@berkeley.edu\\
\textbf{Supplementary information:} Supplementary data are available at
Bioinformatics Online.}

\maketitle

\section{Introduction}

The development of genome browsers is widely considered to be one of the
fundamental milestones of the genomic revolution
\citep{packer2007clickable}. Genome browsers provide researchers the
ability to visually display and explore genomic annotations and data.
Due to their widespread adoption and use, the linear display of genomic
information along reference coordinates is one of the most common
represenations of biological data in the 21st century.

Since their original development during the advent of genome sequencing
\citep{kent2002human, birney2004overview}, genome browsers have made
considerable gains in performance and sophistication. One important
development has been the implementation of genome browsers in
JavaScript, beginning with JBrowse \citep{buels2016jbrowse}. Leveraging
JavaScript makes it possible to move computation that previously took
place on a server into the client browser. Another core advantage of
JavaScript based browsers is that they can leverage modern web
technologies such as Canvas and SVG, providing a more responsive and
interactive experience for the user.

More recently, JBrowse has been rewritten using newer web technologies
such as ReactJS and TypeScript to create an extensible platform for
visualizing and integrating biological data called JBrowse 2. The
platform can be configured and deployed with custom data and settings.
This architecture enables research communities to develop and maintain
curated sets of resources and data on the web, such as WormBase for the
\emph{C. elegans} community \citep{harris2010wormbase}. A new product
offering of JBrowse 2 is a React component that renders a configurable
genome browser, enabling researchers to embed custom browsers into
existing React applications.

While the JBrowse 2 React component is powerful and extensible, it can
present a steep learning curve for bioinformaticians who don't have
experience with React development. On the other hand, the R programming
language and environment is widely used in the bioinformatics community,
as evidenced by the size and and usage of efforts such as Bioconductor
\citep{huber2015orchestrating}. To bridge this gap, we introduce
JBrowseR, an R interface the JBrowse 2 genome browser. JBrowseR is an R
package with functions for embedding a custom browser instance in a
Shiny app, R Markdown document, or the R console.

\section{Materials and Methods}

Here is how to include math equations in the document (bounded by
\texttt{\$\$}):

\[
\begin{aligned}
(x+y)^3&=(x+y)(x+y)^2\\
       &=(x+y)(x^2+2xy+y^2) \label{eqn:example} \\
       &=x^3+3x^2y+3xy^3+x^3. 
\end{aligned}
\]

Describe the approach. Lorem ipsum ad nauseum. Introduce your topic.
Lorem ipsum ad nauseum. Introduce your topic. Lorem ipsum ad nauseum.
Introduce your topic. Lorem ipsum ad nauseum. Introduce your topic.
Lorem ipsum ad nauseum.

Describe the approach. Lorem ipsum ad nauseum. Introduce your topic.
Lorem ipsum ad nauseum. Introduce your topic. Lorem ipsum ad nauseum.
Introduce your topic. Lorem ipsum ad nauseum. Introduce your topic.
Lorem ipsum ad nauseum.

Describe the approach. Lorem ipsum ad nauseum. Introduce your topic.
Lorem ipsum ad nauseum. Introduce your topic. Lorem ipsum ad nauseum.
Introduce your topic. Lorem ipsum ad nauseum. Introduce your topic.
Lorem ipsum ad nauseum.

Describe the approach. Lorem ipsum ad nauseum. Introduce your topic.
Lorem ipsum ad nauseum. Introduce your topic. Lorem ipsum ad nauseum.
Introduce your topic. Lorem ipsum ad nauseum. Introduce your topic.
Lorem ipsum ad nauseum.

\begin{figure}
\centering
\includegraphics{JBrowseR-paper_files/figure-latex/figure-1.pdf}
\caption{Figure from an Rmd chunk.}
\end{figure}

\section{Methods}

Detailed methods. Lorem ipsum ad nauseum. Introduce your topic. Lorem
ipsum ad nauseum. Introduce your topic. Lorem ipsum ad nauseum.
Introduce your topic. Lorem ipsum ad nauseum. Introduce your topic.
Lorem ipsum ad nauseum.

Detailed methods. Lorem ipsum ad nauseum. Introduce your topic. Lorem
ipsum ad nauseum. Introduce your topic. Lorem ipsum ad nauseum.
Introduce your topic. Lorem ipsum ad nauseum. Introduce your topic.
Lorem ipsum ad nauseum.

Detailed methods. Lorem ipsum ad nauseum. Introduce your topic. Lorem
ipsum ad nauseum. Introduce your topic. Lorem ipsum ad nauseum.
Introduce your topic. Lorem ipsum ad nauseum. Introduce your topic.
Lorem ipsum ad nauseum.

\subsection{Sub-Method}

Details for Method 1. Lorem ipsum ad nauseum. Introduce your topic.
Lorem ipsum ad nauseum. Introduce your topic. Lorem ipsum ad nauseum.
Introduce your topic. Lorem ipsum ad nauseum. Introduce your topic.
Lorem ipsum ad nauseum.

Details for Method 1. Lorem ipsum ad nauseum. Introduce your topic.
Lorem ipsum ad nauseum. Introduce your topic. Lorem ipsum ad nauseum.
Introduce your topic. Lorem ipsum ad nauseum. Introduce your topic.
Lorem ipsum ad nauseum.

Details for Method 1. Lorem ipsum ad nauseum. Introduce your topic.
Lorem ipsum ad nauseum. Introduce your topic. Lorem ipsum ad nauseum.
Introduce your topic. Lorem ipsum ad nauseum. Introduce your topic.
Lorem ipsum ad nauseum.

Details for Method 1. Lorem ipsum ad nauseum. Introduce your topic.
Lorem ipsum ad nauseum. Introduce your topic. Lorem ipsum ad nauseum.
Introduce your topic. Lorem ipsum ad nauseum. Introduce your topic.
Lorem ipsum ad nauseum.

\subsection{Method 2}

Details for Method 2. Lorem ipsum ad nauseum. Introduce your topic.
Lorem ipsum ad nauseum. Introduce your topic. Lorem ipsum ad nauseum.
Introduce your topic. Lorem ipsum ad nauseum. Introduce your topic.
Lorem ipsum ad nauseum.

Details for Method 2. Lorem ipsum ad nauseum. Introduce your topic.
Lorem ipsum ad nauseum. Introduce your topic. Lorem ipsum ad nauseum.
Introduce your topic. Lorem ipsum ad nauseum. Introduce your topic.
Lorem ipsum ad nauseum.

Details for Method 2. Lorem ipsum ad nauseum. Introduce your topic.
Lorem ipsum ad nauseum. Introduce your topic. Lorem ipsum ad nauseum.
Introduce your topic. Lorem ipsum ad nauseum. Introduce your topic.
Lorem ipsum ad nauseum.

Details for Method 2. Lorem ipsum ad nauseum. Introduce your topic.
Lorem ipsum ad nauseum. Introduce your topic. Lorem ipsum ad nauseum.
Introduce your topic. Lorem ipsum ad nauseum. Introduce your topic.
Lorem ipsum ad nauseum.

\section{Discussion}

Discussion of results. Lorem ipsum ad nauseum. Introduce your topic.
Lorem ipsum ad nauseum. Introduce your topic. Lorem ipsum ad nauseum.
Introduce your topic. Lorem ipsum ad nauseum. Introduce your topic.
Lorem ipsum ad nauseum.

Discussion of results. Lorem ipsum ad nauseum. Introduce your topic.
Lorem ipsum ad nauseum. Introduce your topic. Lorem ipsum ad nauseum.
Introduce your topic. Lorem ipsum ad nauseum. Introduce your topic.
Lorem ipsum ad nauseum.

Discussion of results. Lorem ipsum ad nauseum. Introduce your topic.
Lorem ipsum ad nauseum. Introduce your topic. Lorem ipsum ad nauseum.
Introduce your topic. Lorem ipsum ad nauseum. Introduce your topic.
Lorem ipsum ad nauseum.

Discussion of results. Lorem ipsum ad nauseum. Introduce your topic.
Lorem ipsum ad nauseum. Introduce your topic. Lorem ipsum ad nauseum.
Introduce your topic. Lorem ipsum ad nauseum. Introduce your topic.
Lorem ipsum ad nauseum.

\section{Conclusion}

Anything else? Lorem ipsum ad nauseum. Introduce your topic. Lorem ipsum
ad nauseum. Introduce your topic. Lorem ipsum ad nauseum. Introduce your
topic. Lorem ipsum ad nauseum. Introduce your topic. Lorem ipsum ad
nauseum.

Anything else? Lorem ipsum ad nauseum. Introduce your topic. Lorem ipsum
ad nauseum. Introduce your topic. Lorem ipsum ad nauseum. Introduce your
topic. Lorem ipsum ad nauseum. Introduce your topic. Lorem ipsum ad
nauseum.

Anything else? Lorem ipsum ad nauseum. Introduce your topic. Lorem ipsum
ad nauseum. Introduce your topic. Lorem ipsum ad nauseum. Introduce your
topic. Lorem ipsum ad nauseum. Introduce your topic. Lorem ipsum ad
nauseum.

Anything else? Lorem ipsum ad nauseum. Introduce your topic. Lorem ipsum
ad nauseum. Introduce your topic. Lorem ipsum ad nauseum. Introduce your
topic. Lorem ipsum ad nauseum. Introduce your topic. Lorem ipsum ad
nauseum.

\section*{Acknowledgements}
\addcontentsline{toc}{section}{Acknowledgements}

These should be included at the end of the text and not in footnotes.
Please ensure you acknowledge all sources of funding, see funding
section below.

Details of all funding sources for the work in question should be given
in a separate section entitled `Funding'. This should appear before the
`Acknowledgements' section.

\section*{Funding}
\addcontentsline{toc}{section}{Funding}

The following rules should be followed:

\begin{itemize}
\tightlist
\item
  The sentence should begin: `This work was supported by \ldots{}' -
\item
  The full official funding agency name should be given, i.e.~`National
  Institutes of Health', not `NIH' (full RIN-approved list of UK funding
  agencies)
\item
  Grant numbers should be given in brackets as follows: `{[}grant number
  xxxx{]}'
\item
  Multiple grant numbers should be separated by a comma as follows:
  `{[}grant numbers xxxx, yyyy{]}'
\item
  Agencies should be separated by a semi-colon (plus `and' before the
  last funding agency)
\item
  Where individuals need to be specified for certain sources of funding
  the following text should be added after the relevant agency or grant
  number `to {[}author initials{]}'.
\end{itemize}

An example is given here: `This work was supported by the National
Institutes of Health {[}AA123456 to C.S., BB765432 to M.H.{]}; and the
Alcohol \& Education Research Council {[}hfygr667789{]}.'

Oxford Journals will deposit all NIH-funded articles in PubMed Central.
See Depositing articles in repositories -- information for authors for
details. Authors must ensure that manuscripts are clearly indicated as
NIH-funded using the guidelines above.


% Bibliography
\bibliographystyle{natbib}
\bibliography{bibliography.bib}

\end{document}
